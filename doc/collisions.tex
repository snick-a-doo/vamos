\documentclass{article}
\usepackage{epsfig}

\newcommand{\mat}[1]{%
  \ensuremath{\mathsf \mathbf #1}}

\newcommand{\skewend}{%
  \ensuremath{\sigma_\mathrm{end}}}

\begin{document}

\title{Collision Calculations in Vamos}
\author{Sam Varner\\ \texttt{snick-a-doo@comcast.net}}
\maketitle

This document describes the way that collisions are handled in Vamos
so I can understand and fix them.

\section {Impulse}
When one object runs into another, it gets an impulse that results in a
change in momentum.  We can treat an impulse as a force that's applied
for a very short time.  In the simulation, we'll apply the force for
one time step.

Starting at the end, the impulse for a two-body collision is given by
\begin{equation}
  \label{eq:2-body-impulse}
  \mathbf{j}=\mathbf{n} \frac{-(1+e) \mathbf{v} \cdot \mathbf{n}}
  {\mathbf{n} \cdot \mathbf{n} (1/m_1 + 1/m_2) 
    + [(\mathbf{I}_1^{-1} (\mathbf{r}_{1p} \times \mathbf{n})) \times \mathbf{r}_{1p}
    + (\mathbf{I}_2^{-1} (\mathbf{r}_{2p} \times \mathbf{n})) \times \mathbf{r}_{2p}]
    \cdot \mathbf{n}}
\end{equation}
Where $e$ is the coefficient of restitution, $\mathbf{v}$ is the relative
velocities of the centers of mass of the two objects, $\mathbf{n}$ is
the normal vector, the $m$s are masses, $\mathbf{I}$s are inertia
tensors, and the $\mathbf{r}$s are vectors from the centers of mass to
the contact point.

For the collision of a body with a fixed object we take the limit of
large $m_2$ and $\mathbf{I}_2$ to get
\begin{equation}
  \label{eq:1-body-impulse}
  \mathbf{j}=\mathbf{n} \frac{-(1+e_1e_2) \mathbf{v} \cdot \mathbf{n}}
  {\mathbf{n} \cdot \mathbf{n} /m_1
    + [(\mathbf{I}_1^{-1} (\mathbf{r} \times \mathbf{n})) \times \mathbf{r}]
    \cdot \mathbf{n}}
\end{equation}
The normal vector is the normal to the surface where contact is made.
I'm guessing that the coefficient of restitution is the product of the
object material's and the surface material's coefficient, but I'm not
convinced that that's correct.  The velocity is the velocity of the
point of contact.  The mass and inertia tensor are calculated from the
particles that make up the object.  The vector $\mathbf{r}$, called
``moment'' in the code, is the vector from the object's center of mass
to the contact point.  All vectors ($\mathbf{n}$, $\mathbf{v}$, and
$\mathbf{r}$) are expressed in world coordinates.

\section{Collision Detection and Response}
One of the design principles of Vamos is that cars know nothing about
tracks and tracks know nothing about cars.  Consequently, there are no
compile-time dependencies of the track code on the car code or vice
versa.  Also the car code can be tested completely independently of
the track code.  If we're going to stick with this organization then
we need some higher-level object to manage the interaction between car
and track.  This is a job of the {\em world}.

The sequence of events for handling a collision goes something like
this:  The world asks the car where its components are.  Then it asks
if any of these positions is inside something solid, like the road or
a wall.  The track responds ``yes'' or ``no''.  If ``yes'' it also
volunteers some information like the normal vector at that point and
the material that the point is in.  For each ``yes'' response from the
track, the world asks for more information about the components, like
its velocity, the vector from the center of mass to component, the
mass of car, and its inertia tensor.  Once it has collected all of
the information it needs, the world calculates the impulse and applies
it to the car.  In short, the world does a lot of sneaking around to
keep the car and track from finding out about each other.

\subsection {Details}

The main event loop calls \verb#GL_World::animate()# each time
through (unless paused) before calling \verb#Gl_World::display()#
to render the scene.  The call to \verb#animate()# advances the
simulation in time.  Here's a simplified outline of the code.
\begin{verbatim}
Gl_World::animate ()
  Gl_World::propagate_cars (time_step)
    car->propagate (time_step)
    World::interact (car, road_index segment_index)
\end{verbatim}
Within \verb#interact()# each particle is checked to see if it has
passed through the track or a barrier by calling
\verb#Strip_Track::test_for_contact# which returns a
\verb#Contact_Info# structure.  The \verb#Contact_Info# structure
tells
\begin{itemize}
\item if contact was made
\item how far the object has penetrated the track or barrier
\item the normal vector at the contact point
\item the material that the object collided with
\end{itemize}
The property item of the material we're interested in right now is
its restitution factor.

We get the moment, velocity, mass, inertia tensor, and the object's
restitution factor from the car that is passed to \verb#interact()#.
Getting the mass, inertia tensor, and restitution factor is
straightforward.  But with the moment and velocity, we must take
special care that we have world-frame vectors.
\end{document}
